
\documentclass[12pt]{amsart}
\usepackage{geometry} % see geometry.pdf on how to lay out the page. There's lots.
\geometry{a4paper} % or letter or a5paper or ... etc
% \geometry{landscape} % rotated page geometry
\usepackage{url}
\usepackage{amsthm}

% See the ``Article customise'' template for come common customisations

   % The amssymb package provides \mathbb and other
   % math symbols.  The amsmath package provides sophisticated math
   % constructions.  The amsthm package provides \theoremstyle and
   % the \proof environment.
   %
   % The amsmath and amsthm packages are automatically activated by
   % \documentclass{amsart}, so there is no need to activate them here.

      \usepackage{amssymb}

   % Next we use \newtheorem to specify our theorem-like environments
   % (theorem, definition, etc.) and how to display and number them.
   %
   % Note: The \theoremstyle declarations affect the appearance of the
   % Theorems, Definitions, etc.

      \theoremstyle{plain}
      \newtheorem{theorem}{Theorem}[section]
      \newtheorem{lemma}[theorem]{Lemma}
      \newtheorem{corollary}[theorem]{Corollary}
      
      \theoremstyle{definition}
      \newtheorem{definition}[theorem]{Definition}
      
      \theoremstyle{remark}
      \newtheorem{remark}[theorem]{Remark}

   % The preamble is also a good place to define new commands and macros.
   % This part of the preamble is strictly optional according to your taste.

      \newcommand{\R}{{\mathbb R}}
      \newcommand{\nil}{\varnothing}

   % The following mysterious maneuver gets rid of AMS junk at the top
   % and bottom of the first page.
   
      \makeatletter
      \def\@setcopyright{}
      \def\serieslogo@{}
      \makeatother


\title{The magnitude of
 electromagnetic time dilation.}
\author{Howard A. Landman}
\date{v3.2\ \today} % delete this line to display the current date

%%% BEGIN DOCUMENT
\begin{document}

\begin{abstract}
Theories unifying gravity and electromagnetism
naturally give rise to the question of whether there might be
a time dilation associated with the electromagnetic 4-potential.
We show here that
the magnitude of EM time dilation
can be computed from elementary considerations
that are independent of specific unified theories.
We further show that the electrostatic part of the effect is well within reach of experiment,
while the magnetic part is not.
%Finally we suggest simple experiments to answer this question.

\end{abstract}

\maketitle

\tableofcontents

\section{Introduction}

From the first publication of General Relativity in 1915 to about 1930, hundreds of classical theories
were proposed attempting to unify gravity and electromagnetism\cite{Vizgin1994}.
While none of these was completely successful, some of them were very influential.
For example, Weyl's Space-Time-Matter theory\cite{Weyl1922} introduced the notion of gauge invariance,
while Kaluza-Klein theory\cite{Kaluza1921,Klein1926} used a compact 5th dimension
and was an important precursor to string theory

Given that there is a time dilation associated with the gravitational potential in GR,
it seems reasonable to wonder whether there might be a similar time dilation
associated with the EM potential in such unified theories.
Sadly, this question has rarely been asked, let alone answered.
Even after nearly a century, we don't know whether Kaluza-Klein theory has this feature or not.
David Apsel in 1978-1981 gave probably the first unified theory to explicitly predict such a time dilation\cite{Apsel1978,Apsel1979,Apsel1981},
and only a handful of subsequent papers\cite{Rodrigues1983,Ryff1985,Beil1987,vanHolten1991,vanHolten1993,Dubey2015,Ogonowski2012,Ogonowski2013} mention anything similar.
At least to first order,
all of these theories agree on the magnitude of EM time dilation.

In this paper we show why they must.
We derive the magnitudes of both gravitational and electromagnetic time dilations
from elementary considerations
that do not depend on the machinery of GR or any specific unified theory,
and thereby demonstrate that they must be features
of any unified theory that is compatible with both Special Relativity and Quantum Mechanics.

\section{History}

Einstein first derived gravitational time dilation in his 1907 paper on the Relativity Principle\cite{Einstein1907}.
He began in \S 18 by using Special Relativity to show that clocks at different X-positions in a reference frame accelerated in the X-direction cannot run at the same rate;
then in \S 19 used the Equivalence Principle to infer that the same thing must be true
for clocks at different values of a gravitational potential $\Phi$.
He carried out all the arguments to first order to give the linear form $T_d = 1 + \Phi/c^2$, which has since become called the weak-field approximation,
although he did note in passing that the actual formula must be
%exponential in form
$T_d = e^{\Phi/c^2}$.
The linear approximation cannot be exactly correct because it has two problems:
it is logically inconsistent since $(1+\Phi/c^2)(1-\Phi/c^2) \ne 1$,
and there is an event horizon at $\Phi = -c^2$.
The exponential form solves both of those problems.
%We emulate his linearized first-order approach in what follows,
%but return later to briefly consider the non-linear form.

The conclusion of the 1907 argument is that {\em{any}} acceleration
causes the rate of time flow to be a function of position
in the direction of the acceleration.
It did not matter to Einstein whether the acceleration was caused by a rocket,
or by standing on the ground in a gravitational field.
%or by being attached to a spinning disk.
Although he didn't mention it,
it is worth noting that the acceleration of a charged particle by an electric field
is not immune to this argument.
Neither are accelerations due to the weak and strong forces;
{\em{all}} accelerations of a given magnitude
{\em{must}} cause exactly the same time dilation.

%In 1911 Einstein revisited this question.
%He used the thought experiment of a mass dropped from a tall tower,
%which gains kinetic energy $mgh = \frac{1}{2}mv^2$ as it falls.
%At the base of the tower,
%the mass and kinetic energy are converted to pure energy $E = mc^2 + mgh$,
%which is then converted to photons and sent back to the top of the tower.
%Clearly, if the energy received at the top of the tower differs from $mc^2$,
%we have violated conservation of energy.
%He concluded that this required time dilation,
%and a perceived gravitational redshift.

%It is difficult to generalize the 1911 argument to an EM field.
%Since photons are uncharged,
%they cannot carry charge back to the top of the tower.
%Thus any charge on the dropped mass would lead to a charge imbalance.

After General Relativity in 1915 and the Schwarzschild solution in 1916,
another view became possible,
although it is still not widely appreciated.
Taking the weak field ($r_s \ll r$) and low speed ($\frac{dr}{dt} \ll c$) limit of the Schwarzschild metric
leaves us with the Newtonian metric
$$ds^2 = (dx^2 + dy^2 +dz^2 - c^2dt^2) + (-\frac{2GM}{rc^2})c^2dt^2$$
which is just flat Minkowski spacetime plus the time dilation field.
In this metric, space is completely flat and only time is curved,
and the curved time gives geodesics that match Newtonian gravity.
This pure time dilation field appears as a $1/r^2$ ``force''.
So in the Newtonian limit of GR,
%``matter tells time how to dilate, and time dilation tells matter how to move''.
matter causes time dilation and time dilation causes gravitational acceleration.
The direction of cause and effect is completely reversed from the 1907 argument.

If we accept both of these arguments,
then we cannot have any acceleration without an associated time dilation gradient,
and we cannot have any time dilation gradient without an associated acceleration.
The two are inextricably linked.

%\section{Gravitational time dilation without General Relativity}
\section{Gravitational time dilation from $E = h\nu$ and $E = mc^2$}

In this section we use a new method
to derive gravitational time dilation without directly invoking relativity theory.
We assume only that particles have a rest energy associated with their mass, given by $E = mc^2$,
and a frequency associated with their energy, given by $E = h\nu$.

In a uniform gravitational field of strength $g$, raising the particle by a height $z$ requires work $mgz$. Thus, to an observer at height $0$, the total energy of the particle at height $z$ is given by $E(z) = mc^2 + mgz$ and its frequency by $\nu(z) = E(z)/h$.

However, an observer already at height $z$ would perceive the particle to have merely frequency $\nu(0) = mc^2/h$. This can only be true if the two observers have clocks running at different rates, in the ratio
\[ T_d = \frac{\nu(z)}{\nu(0)} = \frac{E(z)}{E(0)} = \frac{mc^2 + mgz}{mc^2} = 1 + \frac{gz}{c^2} \]
\noindent which is the weak-field approximation to GR's gravitational time dilation
(with $\Phi = gz$).
As above, the linear form can't be exactly correct but the exponential form
$T_d = e^{gz/c^2}$ is.

This derivation appears in some sense to be quantum, since it utilizes $E = h\nu$.
But because time dilation is a ratio, $h$ cancels out and its precise value doesn't matter.
This means that the classical ($h \to 0$) limit is exactly the same as the ``quantum'' result.

Both this derivation and Einstein's 1907 one avoid almost all the assumptions of GR,
so each of them implies that any other theory that predicts a gravitational time dilation
must have the same relation of dilation to potential as GR.
%as long as $E = mc^2$ and $E = h\nu$ are both still true.
From this viewpoint, the existence and magnitude of gravitational time dilation
cannot be viewed as a confirmation of GR specifically,
but only of a class of theories of which GR is the best known example.

%There is a curious paradox hidden in the above results.
%$\nu = E/h$ implies that in QM the (quantum phase) frequency of things is linearly proportional to their energy.
%$T_d = e^{ gz/c^2 } = e^{ mgz/mc^2 }$ implies that in GR
%the (time-dilated) frequency of things is an exponential function of their energy.
%Both formulae are widely accepted, but
%it seems obvious that they cannot both be universally true.
%Reconciling this apparent contradiction is beyond the scope of the current paper.

\section{Electromagnetic time dilation by the same method}

We now consider the case of a particle with mass $m$ and charge $q$ in an electrostatic potential $V$.
The potential energy is $qV$, so the corresponding time dilation (to first order) must be
\[ T_d =  \frac{mc^2 + qV}{mc^2} =  1 + \frac{qV}{mc^2} \]
As in the gravitational case,
the linear form cannot be completely right, and
the exponential form 
$T_d = e^{qV/mc^2}$
seems the most obvious candidate to replace it.
But unlike in the gravitational case, here both charge and mass matter;
this means that a simple Riemannian manifold is inadequate,
and the geometry of any unified theory would have to be something more complicated,
like a Finsler space.
Uncharged particles should be completely unaffected.
For a given non-zero $q$, lighter particles will be dilated more strongly than heavier particles.
The electron, being the lightest charged particle and having the highest charge-to-mass ratio,
should be affected the most.
But since electrons have infinite lifetime,
the only observable effect on them is the shift in phase frequency.
Although this is universally observed,
most physicists would not consider it proof of or even evidence for time dilation.

Thus, for experimental testing, we are lead to the muon.
With a mass-energy of $m_\mu c^2 = 105.7\ \textup{MeV}$,
it is still light enough to have its mean lifetime of $2.2\ \mu \textup{S}$
affected by a modest potential.
For example, a potential of 1.057 MV should alter its lifetime by about 1\%;
such a potential could be achieved by a Van de Graaff generator
with a sphere of about 76 cm diameter in air,
which is well within reach of a serious hobbyist.
Apsel first proposed this kind of experiment in 1979\cite{Apsel1979};
40 years later it still has never been performed.

Muon bound to low-Z nuclei are also known to have lengthened lifetimes.
The normal explanation for this is that the muon has a kinetic energy given by the quantum virial theorem,
and an average velocity corresponding to that kinetic energy,
and a special-relativistic time dilation corresponding to that velocity.
However, Apsel has argued that this calculation does not match the experimental data very well,
and that adding electromagnetic time dilation gives a better fit\cite{Apsel1981}.
If so, we may have already been seeing evidence for decades.
The effect should be more obvious for higher-Z.
Unfortunately, as Z increases, nuclear capture by a proton begins to dominate,
and we don't have good data on non-capture decay rates for most elements.

For magnetic interactions, the muon's measured magnetic moment is
$-4.49 \times 10^{-26}\ \textup{J}/\textup{T}$.
To get the same 1\% level of time dilation, say between spin-up and spin-down muons,
we would need to place them in a field of

\[ 1.057\ \textup{MeV} \times \frac{1\ \textup{J}}{6.24\times 10^{12}\ \textup{MeV}} \times \frac{1\ \textup{T}}{2 \times (4.49 \times 10^{-26}\ \textup{J})} = 1.89 \times 10^{12}\ \textup{T} \]

Given that the world record magnetic fields are in the range of 45--330 T,
this seems beyond the reach of current experiment.
Van Holten thought that $5 \times 10^9\ \textup{T}$  might suffice for detection
and could be found in the vicinity of a magnetar\cite{vanHolten1991,vanHolten1993}.

One characteristic of a pure time dilation is that, all other things being equal,
it must necessarily slow down (or speed up) all decay modes equally.
Since muons have 3 known decay modes\cite{PDG2014}, this can be used as a test
for whether lifetime alterations can reasonably be viewed as solely due to time dilation,
or whether other factors must be invoked.


\section{Counterarguments}

There are many counterarguments to EM time dilation theories. In this section we point out flaws in two of the main ones.

\subsection{Naive Electromagnetic Gauge Invariance}

Many physical theories, such as classical EM and Van Holten's theory mentioned in the previous section,
have a property that I will call Naive Electromagnetic Gauge Invariance.
In NEGI theories, everything can be expressed in terms of fields acting locally, and
potentials can be viewed as having no physical reality but being merely aids to computation.
NEGI would of course rule out any time dilation effects from an EM potential in a field free region,
such as inside the sphere of a Van De Graaff generator.
Many physicists seem to think that this is sufficient to disprove the theory.

%The standard counterargument is that the gauge invariance of most modern theories makes the proposed effect impossible.
The problem with this viewpoint is that it is flat-out wrong.
The universe does {\em not} have the NEGI property;
the Aharonov-Bohm effect\cite{Ehrenberg1949,Aharonov1959} suffices as a counterexample.
The importance of this is often glossed over.
For example, Jackson and Okun\cite[p.24]{Jackson2001} write:
\begin{quote}
\ldots\ gauge invariance is a manifestation of non-observability of $A_\mu$.
However integrals \ldots\ are observable when they are taken over a closed path, as in the Aharonov-Bohm effect \ldots\@
The loop integral of the vector potential there can be converted by Stokes's theorem into the magnetic flux through the loop,
showing that the result is expressible in terms of the magnetic field, albeit in a nonlocal manner.
It is a matter of choice whether one wishes to stress the field or the potential, but the local vector potential is not an observable.
\end{quote}
Contrast this with the discussion in Feynman Vol. II\cite{FeynmanII} lecture 15-5,
where the central importance of the potential is emphasized:
\begin{quote}
The fact that the vector potential appears in the wave equation of quantum mechanics
(called the Schr\"{o}dinger equation)
was obvious from the day it was written.
That it cannot be replaced by the magnetic field in any easy way
was observed by one man after the other who tried to do so.
This is also clear from our example
of electrons moving in a region where there is no field and being affected nevertheless.
But because in classical mechanics \textit{\textbf{A}} did not appear to have any direct importance and,
furthermore, because it could be changed by adding a gradient,
people repeatedly said that the vector potential had no direct physical significance ---
that only the magnetic and electric fields are ``right'' even in quantum mechanics.
It seems strange in retrospect that no one thought of discussing this experiment until 1956 \ldots
The implication was there all the time, but no one paid attention to it. \ldots\
It is interesting that something like this can be around for thirty years but,
because of certain prejudices of what is and is not significant,
continues to be ignored.
\end{quote}
In either case, the NEGI idea (that fields acting locally can explain everything) is admitted to be false.

\subsection{CPT Invariance}

It is often stated (e.g. in \cite{Andreev2007,Murayama2003,Sachs1987}) that the CPT theorem guarantees that particle and antiparticle masses and lifetimes are identical.
However, this conclusion is only justified at zero potential,
or with the further assumption of NEGI (which renders potential irrelevant).
A true CPT reflection must invert all charges and magnetic moments in the universe,
which necessarily inverts all EM potentials as well.
Therefore, the CPT theorem only {\em really} proves that a particle's mass and lifetime at 4-potential \textit{\textbf{A}} must equal its antiparticle's mass and lifetime at 4-potential -\textit{\textbf{A}}.
This holds true under EM time dilation, since the dilations for those two cases are identical.
Thus, the CPT theorem does not contradict the claim that particles and antiparticles will be time-dilated oppositely at a non-zero potential and that their lifetimes will differ there.
EM time dilation is completely compatible with the notion of CPT invariance.


\section{Summary}

We reviewed two early derivations of gravitational time dilation
and gave a new elementary derivation of it.
Both the 1907 Einstein derivation and this new method
can be trivially modified to give derivations of electromagnetic time dilation as well,
which agree in magnitude with the handful of prior theories predicting such an effect.
That EM time dilation seems so inescapably implied,
and is yet so widely rejected,
%(with multiple counter-arguments, albeit some flawed)
points perhaps to a deep paradox in current physical thought.
Since testing for the first-order electrostatic effect would be quite easy and cheap,
it seems worthwhile to actually perform that experiment.


\begin{thebibliography}{10}

\bibitem{Vizgin1994}
V.P. Vizgin (tr. J.B. Barbour),
``Unified Field Theories in the first third of the 20th century'',
Birkh\"{a}user Verlag (1994)

\bibitem{Weyl1922}
H. Weyl,
{\em Space-Time-Matter} (1922)

\bibitem{Kaluza1921}
T. Kaluza,
``Zum Unit\"{a}tsproblem in der Physik'',
{\em Sitzungsber. Preuss. Akad. Wiss. Berlin. (Math. Phys.)}: 966�972. (1921)
 
\bibitem{Klein1926}
O. Klein,
``Quantentheorie und f\"{u}nfdimensionale Relativit\"{a}tstheorie'',
{\em Zeitschrift f�r Physik A.} 37 (12): 895�906 (1926)
doi:10.1007/BF01397481.

\bibitem{Apsel1978}
D. Apsel,
``Gravitational, electromagnetic, and nuclear theory'',
{\em 	International Journal of Theoretical Physics} v.17 \#8 643-649 (Aug 1978)
DOI: 	10.1007/BF00673015

\bibitem{Apsel1979}
D. Apsel,
``Gravitation and electromagnetism'',
{\em General Relativity and Gravitation} v.10 \#4 297-306 (Mar 1979)
DOI: 10.1007/BF00759487

\bibitem{Apsel1981}
D. Apsel,
``Time dilations in bound muon decay'',
{\em General Relativity and Gravitation} v.13 \#6 605-607 (Jun 1981)
DOI: 10.1007/BF00757247

\bibitem{Rodrigues1983}
W.A. Rodrigues Jr.,
``The Standard of Length in the Theory of Relativity and Ehrenfest Paradox'',
{\em Il Nuovo Cimento} v.74 B \#2 199-211 (11 April 1983)

\bibitem{Ryff1985}
L.C.B. Ryff,
``The Lifetime of an Elementary Particle in a Field'',
{\em General Relativity and Gravitation} v.17 \#6 515-519 (1985)

\bibitem{Beil1987}
R.G. Beil,
``Electrodynamics from a Metric'',
{\em Int. J. of Theoretical Physics} v.26 \#2 189-197 (1987)

\bibitem{vanHolten1991}
J.W. van Holten,
``Relativistic Time Dilation in an External Field'',
NIKHEF-H/91-05 (1991)

\bibitem{vanHolten1993}
J.W. van Holten,
``Relativistic Dynamics of Spin in Strong External Fields'',
\url{arXiv:hep-th/9303124v1}, (24 March 1993)

\bibitem{Dubey2015}
A.K. Dubey, A.K. Sen,
``Gravitational Redshift in Kerr-Newman Geometry'',
\url{arXiv:1503.03833v5}, (16 October 2015)

%\bibitem{Creator2004}
%``time dilation in an electromagnetic potential'' (2004)\\
%\url{http://www.physicsforums.com/archive/index.php/t-57510.html}

%\bibitem{Duda2010}
%\url{http://groups.google.com/group/sci.physics.relativity/browse_thread/thread/3be6a489686aed86/}

\bibitem{Ogonowski2012}
P. Ogonowski,
``Time dilation as field'',
Journal of Modern Physics, 3, 200-207 (2012)

\bibitem{Ogonowski2013}
P. Ogonowski, P. Skindzier,
``Maxwell-like picture of General Relativity and its Planck limit'',
\url{https://arxiv.org/pdf/1301.2758.pdf}

\bibitem{Einstein1907}
A. Einstein, ``\"{U}ber das Relativit\"{a}tsprinzip und die aus demselben gezogenenn Folgerungen'',
{\em Jahrbuch der Radioaktivit\"{a}t und Elektronik} 4, 411-462 (4 December 1907);
English translation, ``On the relativity principle and the conclusions drawn from it'',
{\em The Collected Papers}, v.2, 433-484 (1989),
available at \url{https://einsteinpapers.press.princeton.edu/vol2-trans/319}

%\bibitem{Diepold2013}
%M. Diepold et al.,
%``Lifetime and population of the 2S state in muonic hydrogen and deuterium'',
%Phys. Rev. A 88, 042520 (31 October 2013)

%\bibitem{Schrodinger1950}
%E. Schr\"{o}dinger, 
%{\em Space-Time Structure} pp.1-2,
%Cambridge Univ. Press (1950)

%\bibitem{Carroll2001}
%S.M. Carroll.
%``A No-Nonsense Introduction to General Relativity'',
%(2001)
%\url{http://preposterousuniverse.com/grnotes/grtinypdf.pdf}

%\bibitem{Hamza2015}
%H.G.A.I. Hamza \& W.A.A.  Alhassan,
%``Deriving the Useful Expression for Time Dilation in the Presence Of the Gravitation by means of a Light Clock'',
%IOSR Journal of Applied Physics, e-ISSN: 2278-4861.Volume 7, Issue 2 Ver. II (Mar.-Apr. 2015), pp.107-111

%\bibitem{TongGR}
%David Tong,
%``Concepts in Theoretical Physics, Lecture 7: General Relativity'',
%\url{http://www.damtp.cam.ac.uk/user/tong/concepts/gr.pdf}

%\bibitem{Nordebo2016}
%``The Reissner-Nordstr�m metric'',
%\url{http://www.diva-portal.org/smash/get/diva2:912393/FULLTEXT01.pdf},
%Section 4.4 pp.22-24

%\bibitem{TerzicPHYS652}
%Bal\v{s}a Terzi\'{c},
%``PHYS 652: Astrophysics'',
%\url{http://nicadd.niu.edu/~bterzic/PHYS652/PHYS652_notes.pdf}
%pp.9-10

%\bibitem{Lawden1982}
%D.F. Lawden, {\em An Introduction to Tensor Calculus, Relativity and Cosmology} 3rd Ed, J.Wiley (1982)

% Rindler Horizon
%\bibitem{wikiRindler}
%``Rindler Coordinates''
%\url{http://en.wikipedia.org/wiki/Rindler_coordinates}

%\bibitem{Egan2006}
%G. Egan,
%``The Rindler Horizon'',
%\url{http://www.gregegan.net/SCIENCE/Rindler/RindlerHorizon.html}

%\bibitem{Rindler2001}
%W. Rindler,
%{\em Relativity: Special, General, and Cosmological}, Oxford U. Press (2001)

%\bibitem{MTW}
%Misner, Thorne, Wheeler,
%{\em Gravitation}

%\bibitem{deBroglie1925}
%L. de Broglie,
%{\em Recherches sur la Th\'{e}orie des Quanta} (1925),
%translated by A.F. Kracklauer as
%{\em On the Theory of Quanta} (2004)
%\url{http://www.ensmp.fr/aflb/LDB-oeuvres/De_Broglie_Kracklauer.pdf}

%\bibitem{Hestenes2008}
%D. Hestenes, ``Electron time, mass and zitter'' (2008)
%\url{http://www.fqxi.org/data/essay-contest-files/Hestenes_Electron_time_essa.pdf}

%\bibitem{Schrodinger1925}
%E. Schr\"{o}dinger, letter to W. Wien (27 Dec 1925)

%\bibitem{Mehra2000}
%J. Mehra \& H. Rechenberg,
%{\em The Historical Development of Quantum Theory: Volume 5, Erwin Schrodinger and the Rise of Wave Mechanics: Part 2, The Creation of Wave Mechanics: Early Response and Applications 1925-1926} 461-465,
%Springer Verlag (Dec 2000)

\bibitem{Ehrenberg1949}
W. Ehrenberg, R. E. Siday,.
``The Refractive Index in Electron Optics and the Principles of Dynamics'',
{\em Proc. Phys. Soc.} B62: 8-21 (1949).
DOI: 10.1088/0370-1301/62/1/303

\bibitem{Aharonov1959}
Y. Aharonov, D. Bohm,
``Significance of Electromagnetic Potentials in the Quantum Theory'',
{\em Phys. Rev.} 115, 485-491 (1959). 
\url{http://link.aps.org/doi/10.1103/PhysRev.115.485}
DOI: 10.1103/PhysRev.115.485

\bibitem{Jackson2001}
J. D. Jackson \& L. B. Okun,
``Historical roots of gauge invariance'',
LBNL-47066 (12 March 2001)

\bibitem{FeynmanII}
R. P. Feynman,
{\em Lectures On Physics} v.II,
Addison Wesley (1964)

\bibitem{Andreev2007}
V.A. Andreev et al.,
``Measurement of the Rate of Muon Capture in Hydrogen Gas and 
Determination of the Proton�s Pseudoscalar Coupling $g_P$'',
submitted to Phys.Rev.Lett
arXiv:0704.2072v1

\bibitem{Murayama2003}
H. Murayama,
``CPT Tests: Kaon vs Neutrinos'',
arXiv:hep-ph/0307127

\bibitem{Sachs1987}
R.G. Sachs,
{\em The Physics of Time Reversal},
U. Chicago Press (1987),
p.175

\bibitem{PDG2014}
K.A. Olive et al. (Particle Data Group),
Chin. Phys. C38, 090001 (2014)
\url{http://pdg.lbl.gov/2014/listings/rpp2014-list-muon.pdf}

%\bibitem{Schwartz1977}
%H.M. Schwartz, ``Einstein's comprehensive 1907 essay on relativity, part 1'',
%{\em Am. J. Physics} 45, 512-517 (1977)

%\bibitem{Jammer2006}
%M. Jammer, {\em Concepts of Simultaneity: from Antiquity to Einstein and Beyond}, Johns Hopkins U. Press (2006)

%\bibitem{Tonomura1982}
%A. Tonomura et al.,
%``Observation of Aharonov-Bohm Effect by Electron Holography'',
%{\em Phys. Rev. Lett.} 48 1443-1446 (1982)

\end{thebibliography}

\end{document}